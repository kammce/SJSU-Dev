\hypertarget{mesh__design_8h}{}\section{/var/www/html/\+S\+J\+S\+U-\/\+D\+E\+V-\/\+Linux/firmware/default/lib/\+L4\+\_\+\+I\+O/wireless/src/mesh\+\_\+design.h File Reference}
\label{mesh__design_8h}\index{/var/www/html/\+S\+J\+S\+U-\/\+D\+E\+V-\/\+Linux/firmware/default/lib/\+L4\+\_\+\+I\+O/wireless/src/mesh\+\_\+design.\+h@{/var/www/html/\+S\+J\+S\+U-\/\+D\+E\+V-\/\+Linux/firmware/default/lib/\+L4\+\_\+\+I\+O/wireless/src/mesh\+\_\+design.\+h}}


This is a documentation only file for the mesh algorithm.  




\subsection{Detailed Description}
This is a documentation only file for the mesh algorithm. 

/$\ast$ Social\+Ledge.\+com -\/ Copyright (C) 2013

This file is part of free software framework for embedded processors. You can use it and/or distribute it as long as this copyright header remains unmodified. The code is free for personal use and requires permission to use in a commercial product.

T\+H\+IS S\+O\+F\+T\+W\+A\+RE IS P\+R\+O\+V\+I\+D\+ED \char`\"{}\+A\+S I\+S\char`\"{}. NO W\+A\+R\+R\+A\+N\+T\+I\+ES, W\+H\+E\+T\+H\+ER E\+X\+P\+R\+E\+SS, I\+M\+P\+L\+I\+ED OR S\+T\+A\+T\+U\+T\+O\+RY, I\+N\+C\+L\+U\+D\+I\+NG, B\+UT N\+OT L\+I\+M\+I\+T\+ED TO, I\+M\+P\+L\+I\+ED W\+A\+R\+R\+A\+N\+T\+I\+ES OF M\+E\+R\+C\+H\+A\+N\+T\+A\+B\+I\+L\+I\+TY A\+ND F\+I\+T\+N\+E\+SS F\+OR A P\+A\+R\+T\+I\+C\+U\+L\+AR P\+U\+R\+P\+O\+SE A\+P\+P\+LY TO T\+H\+IS S\+O\+F\+T\+W\+A\+RE. I S\+H\+A\+LL N\+OT, IN A\+NY C\+I\+R\+C\+U\+M\+S\+T\+A\+N\+C\+ES, BE L\+I\+A\+B\+LE F\+OR S\+P\+E\+C\+I\+AL, I\+N\+C\+I\+D\+E\+N\+T\+AL, OR C\+O\+N\+S\+E\+Q\+U\+E\+N\+T\+I\+AL D\+A\+M\+A\+G\+ES, F\+OR A\+NY R\+E\+A\+S\+ON W\+H\+A\+T\+S\+O\+E\+V\+ER.

You can reach the author of this software at \+: p r e e t . w i k i @ g m a i l . c o m

\begin{DoxyParagraph}{Features}
This mesh algorithm features \+:
\begin{DoxyItemize}
\item Addressed nodes with auto-\/retries, and auto-\/acknowledge.
\item Minimal R\+AM footprint with no H\+E\+AP usage.
\item Each node contains routing table and can act like a repeater node.
\item Each packet contains its retry counts, and maximum hops it can take.
\end{DoxyItemize}
\end{DoxyParagraph}
Detailed features \+:
\begin{DoxyItemize}
\item Each packet sent can use existing route, and if route has changed, a new route is automatically discovered using a special retry packet.
\item Each node\textquotesingle{}s A\+CK contains some piggy-\/back data about the node itself.
\item Duplicate packets are absorbed but an A\+CK is still replied if its a duplicate, but a retry packet.
\item An A\+CK packet or a response to an A\+CK all use retries; even the repeating nodes participate to make sure the packet is delivered.
\end{DoxyItemize}

\begin{DoxyParagraph}{Example use case}
Suppose there are 4 nodes in your network, and the nodes only know about their neighbors (ie\+: N1 only knows N2, but doesn\textquotesingle{}t know where is N3 and N4)\+: \begin{DoxyVerb} N1      N2      N3      N4
\end{DoxyVerb}

\end{DoxyParagraph}
Each node maintains a routing table of 3 fields \+:
\begin{DoxyItemize}
\item Destination Node
\item Next Destination
\item Hop count to destination node
\end{DoxyItemize}

Here is how the mesh algorithm works step-\/by-\/step when N1 sends data to N4 \+: D=Destination, S=Source, MD=Mesh Destination, MS=Mesh Source Table\+: X$\vert$\+Y$\vert$Z where X=TO, Y=Next destination, Z=Hops 

 \subsection*{Here is how a packet is sent to unknown node N4 }

N1 N2 N3 N4 Route Table \+: $<$\+Empty$>$ $<$\+Empty$>$ $<$\+Empty$>$ $<$\+Empty$>$ Packet \+: D(4), S(1) M\+D(0),M\+S(1)

N1 N2 N3 N4 Route Table \+: $<$\+Empty$>$ D1$\vert$\+D1$\vert$0 $<$\+Empty$>$ $<$\+Empty$>$ Packet \+: D(4), S(1) M\+D(0),M\+S(2)

N1 N2 N3 N4 Route Table \+: $<$\+Empty$>$ D1$\vert$\+D1$\vert$0 D2$\vert$\+D2$\vert$0 $<$\+Empty$>$ D1$\vert$\+D2$\vert$1 Packet \+: D(4), S(1) M\+D(0),M\+S(3)

N1 N2 N3 N4 Route Table \+: $<$\+Empty$>$ D1$\vert$\+D1$\vert$0 D2$\vert$\+D2$\vert$0 D3$\vert$\+D3$\vert$0 D1$\vert$\+D2$\vert$1 D1$\vert$\+D3$\vert$2 Packet \+: D(4), S(1) M\+D(0),M\+S(3) 

 \subsection*{Now let\textquotesingle{}s take a look at how the packet travels back \+: }

T\+O\+DO



 Future work and improvement possibilities \+:
\begin{DoxyItemize}
\item R\+SS (Received Signal Strength) should be factored into the routing table for those radios that support it.
\item If routing table is full, the route least used should be removed, therefore we need to add a \char`\"{}score\char`\"{} variable to the routing table structure.
\item Implement \char`\"{}packet\+\_\+status\char`\"{} for each packet sent? ie\+: \char`\"{}\+B\+U\+S\+Y\char`\"{}, \char`\"{}\+R\+E\+T\+R\+Y\+I\+N\+G\char`\"{}, \char`\"{}\+D\+E\+L\+I\+V\+E\+R\+E\+D\char`\"{}, \char`\"{}\+T\+I\+M\+E\+O\+U\+T\char`\"{}, or \char`\"{}\+E\+R\+R\+O\+R\char`\"{}
\item Implement special packet to query N-\/th routing entry of a node.
\item An A\+CK packet should contain what sequence number the A\+CK is for \subsection*{such that more than one packet with an A\+CK can be out at a time. }
\end{DoxyItemize}

How some scenarios are handled \+:

Scenario 1 \+: N1 --$>$ N2 --$>$ N3 N1 knows N3, but what if N3 disappears, and is now directly reachable without any intermediate node ? N1 will send packet with MD being N2, but instead, N3 will directly respond back, and now N1 will update its routing table.

Scenario 2 \+: N1 --$>$ N2 --$>$ N3 N1 knows N3, but N2 has disappeared and a new node N2b has replaced N2. N1 will retry its packets, and eventually realize after retry count-\/down that N3 no longer has a good route entry, so that route entry will be removed, and when N1 resends the packet again, new route discovery will be performed through N2b.

Scenario 3 \+: N1 --$>$ N2 --$>$ N3 N1 knows N3, but say that N2 can no longer reach N3. After some retries, N2 will give up on N3 destined packet because N3 would not have sent an A\+CK back. Option 1 (not implemented) \+: N2 can send a special packet to N1 telling it to remove its route of N3 through N2. Option 2 (simpler) \+: N1 will send retries through N2, and after N3 is not reachable, its route will be removed, and when a packet is sent again, new route discovery will be performed. 

